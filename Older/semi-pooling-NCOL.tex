%!TEX TS-program = xelatex
%!TEX encoding = UTF-8 Unicode
\documentclass[12pt]{article} 
\usepackage{tikz}

%\usepackage{xcolor}
%\usepackage{hyperref}
%\usepackage{subfig}
%\usepackage{xltxtra} ß
\usepackage{enumerate}

%\synctex=1
\usepackage{amsmath,amsthm,amssymb,amsfonts} 
\usepackage{graphicx} 
\usepackage[center]{caption} 
\usepackage{enumerate} 
\usepackage[shortlabels]{enumitem} 
\usepackage[obeyDraft,colorinlistoftodos]{todonotes} 
\usepackage[onehalfspacing]{setspace}

%\usepackage{fontspec}
%\usepackage{xunicode}
%\usepackage{xltxtra} \synctex=1 
\usepackage{natbib} 
\usepackage{subcaption} 
\usepackage[onehalfspacing]{setspace}

%\usepackage[round]{natbib}
%\usepackage[all]{xy}
%\setcounter{MaxMatrixCols}{10}
%
\renewcommand{\l}{\ell}
\newtheorem{theorem}{Theorem}
\newtheorem{acknowledgement}{Acknowledgement}
\newtheorem{algorithm}{Algorithm}
\newtheorem{assumption}{Assumption}
\newtheorem{axiom}{Axiom}
\newtheorem{case}{Case}
\newtheorem{claim}{Claim}
\newtheorem{conclusion}{Conclusion}
\newtheorem{condition}{Condition}
\newtheorem{conjecture}{Conjecture}
\newtheorem{corollary}{Corollary}
\newtheorem{criterion}{Criterion}
\newtheorem{example}{Example}
\newtheorem{exercise}{Exercise}
\newtheorem{lemma}{Lemma}
\newtheorem{proposition}{Proposition} \theoremstyle{definition}
\newtheorem{definition}{Definition}
\newtheorem{notation}{Notation}
\newtheorem{problem}{Problem} \theoremstyle{remark}
\newtheorem{remark}{Remark}
\newtheorem{fact}{Fact}
\newtheorem{solution}{Solution}
\newtheorem{summary}{Summary}
\newtheorem{thm}{Theorem}[section]
\newtheorem{lem}[thm]{Lemma}
\newtheorem{prop}[thm]{Proposition}
\newtheorem{cor}[thm]{Corollary} 


\begin{document} 
\title{Standard Signalling (Without Cost of Lying)} 
\author{Patrick Legros}
\date{\today} 
\maketitle


Players have equiprobable types $C,D$ with payoff structures
\begin{table}
	[!htbp] \centering 
	\begin{tabular}
		{c c c} {\small{Own$\backslash$} Opponent} & $H$ & $L$ \\
		\cline{2-3} $H$ & $50$& $10$\\
		\cline{2-3} $L$ & $24$& $20$ \\
		\cline{2-3}\\
		\multicolumn{3}{c}{Type $C$} 
	\end{tabular}
	\hspace{5em} 
	\begin{tabular}
		{c c c} {\small{Own$\backslash$} Opponent} & $H$ & $L$ \\
		\cline{2-3} $H$ & $22$& $10$\\
		\cline{2-3} $L$ & $24$& $20$ \\
		\cline{2-3}\\
		\multicolumn{3}{c}{Type $D$} 
	\end{tabular}
	\caption{Regime NC} 
\label{tbl:NC} 
\end{table}

\section{No Communication (NC)}
Types $D$ play $L$. Let $x$ be the probability that type $C$ plays $H$. Then $x>0$ when
\[
	\frac{1}{2}(10+(50-10)x)+\frac{1}{2}(10)\geq \frac{1}{2}(20+4 x)+\frac{1}{2}(20)
\]
that is when\[
x\geq \frac{5}{9}
	\]
Hence, either $x=0$, or $x=1$ or $x=\frac{5}{9}$.

\section{Communication without Cost (CWC)}
It is clear that if $C$ communicates with probability one, $D$ will communicate. Indeed, if $D$ does not communicate, in state $(1,0)$ types $C$ plays $L$ with  probability one and $D$ has a payoff of $20$ by not communicating. By communicating, $D$ has a payoff of $\frac{1}{2}(20+4x(1,1))+\frac{1}{2}(20)$, larger than $20$ if $x(1,1)>0$. Hence if types $C$ communicate with probability one, either they do not play $H$ or if they do types $D$ also communicate. This situation is observationally equivalent to the previous regime of no communication.

Another possibility is when types $C$ and $D$ play a mixed strategy in communication. Let $\alpha$ the strategy of $C$ and $\beta$ the strategy of $D$ in communication. If the opponent has communicated, an agent believes that she faces a type $C$ with probability $\frac{\alpha}{\alpha+\beta}$. If the opponent has not communicated, an agent believes that she is facing a type $C$ with probability $\frac{1-\alpha}{2-\alpha-\beta}$.

We will be looking at equilibria consistent with types $C$ playing $H$ with positive probability, and both types communicating, that is $\alpha\beta>0$. 

\subsection{Continuation Strategies}

\paragraph{State $(1,1)$.} It is optimal to set $x(1,1)>0$ if $\frac{\alpha}{\alpha+\beta}(10+40x(1,1))+\frac{\beta}{\alpha+\beta}10$ is greater than $\frac{\alpha}{\alpha+\beta}(20+4x(1,1))+\frac{\beta}{\alpha+\beta}20$, that is 
\begin{equation}\label{CWC-BR11}
	x(1,1)>0 \Leftrightarrow x(1,1)\geq \frac{5}{18}\frac{\alpha+\beta}{\alpha}.
\end{equation}

A necessary condition is that $\alpha\geq \frac{5}{18}\beta$.

\paragraph{State $(0,1)$.} A non-communicating type $C$ facing a communicating opponent plays $H$ when (the belief structure is the same as in state $(1,1)$ but the players expects a communicating type $C$ to play $x(1,0)$),
\begin{equation}\label{CWC-BR01}
	x(1,0)\geq \frac{5}{18}\frac{\alpha+\beta}{\alpha}
\end{equation}
%
and the necessary condition is the same as for having $x(1,1)>0$.

\paragraph{State $(1,0)$.} An opponent of type $C$ plays $x(0,1)$, and therefore playing $H$ is optimal for a type $C$ who communicates when
\[
(1-\alpha)(10+40x(0,1))+(1-\beta) 10 \geq (1-\alpha)(20+4x(0,1))+(1-\beta) 20,
\]
\begin{equation}\label{CWC-BR10}
	x(0,1)\geq \frac{5}{18}\frac{2-\alpha-\beta}{1-\alpha}
\end{equation}

and a necessary condition is that 
\[
1-\alpha\geq \frac{5}{18}(1-\beta)
\]

%
\paragraph{State $(0,0)$.} The belief structure is the same as that of a communicating type $C$ in state $(1,0)$, and therefore $x(0,0)>0$ when
\begin{equation}\label{CWC-BR00}
	x(0,0)\geq \frac{5}{18}\frac{2-\alpha-\beta}{1-\alpha}
\end{equation}
with the same necessary condtion as in state $(1,0)$.

It is possible to have all $x(i,j)>0$ when the bounds in \eqref{CWC-BR10} and \eqref{CWC-BR00} are not greater than one, that is when
\begin{equation}\label{xij>0-CWC}
	\frac{5}{13}\beta\leq \alpha \leq \frac{5}{13}\beta+\frac{8}{13}.
\end{equation}
%
\subsection{Incentives to Communicate}
It is possible that $x(i,j)>0$ for each state $(i,j)$. The necessary condition is \eqref{xij>0-CWC}. In this case, types $C$ weakly prefer to communicate and when
\begin{equation}\label{IC-comm-CWC}
	\begin{split}
	\frac{\alpha}{2}& (10+40x(1,1))+\frac{1-\alpha}{2}(10+40x(0,1))+\frac{1}{2}10\\ 
	&\geq \frac{\alpha}{2} (10+40x(1,0))+\frac{1-\alpha}{2}(10+40x(0,0))+\frac{1}{2}10
	\end{split}
\end{equation}
Note that the condition reduces to
\begin{equation}\label{CWC-NC-a,b>0}
	\alpha (x(1,1)-x(1,0))+(1-\alpha)(x(0,1)-x(0,0))\geq 0.
\end{equation}
%
If this condition binds, it also implies that type $D$ is indifferent between communicating and not communicating. If there is a strict inequality, types $D$ communicate with probability one.



\paragraph{``Calibration''}
In the experiment, we have $\alpha=82\%$ and $\beta=43\%$: these values satisfy the necessary condition \eqref{xij>0-CWC}. Because $\frac{\alpha+\beta}{\alpha}=\frac{125}{82}$ and $\frac{2-\alpha-\beta}{1-\alpha}= \frac{85}{18}$. Therefore, $x(0,0)>0$  requires $x(0,0)>\frac{425}{324}$, clearly absurd. Hence $x(0,0)=0$. Similarly, $x(1,0)>0$ requires that $x(0,1)\geq \frac{425}{324}$, which is also absurd. Hence, $x(i,j)=0$ for all $(i,j)\neq (1,1)$, in contradiction with the experimental results.
% (fold)

\section{Costly Communication}
When a fee of $3$ has to be paid in order to communicate, types $D$ cannot benefit from communicating. Indeed, the gain from communication is equal to $2\alpha(x(1,1)-x(1,0))+2(1-\alpha)(x(0,1)-x(0,0)) $ which is less than $2$ and cannot compensate for the exogenous cost of $3$ of sending a message.

Therefore $\beta=0$. The continuation strategies of type $C$ are as in the previous case. However the incentive of type $C$ to communicate is different. 

If $x(i,j)=0$ for $(i,j)\neq (1,1)$, we need
\[
20 \alpha x(1,1)\geq 3,
\]
hence a minimum value of $\alpha$ of $15\%$ when $x(1,1)=1$. If $x(1,1)=\frac{5}{18})$, the condition is that $\alpha \geq 54\%$.

If now $x(i,j)>0$ for all $(i,j)\neq (1,1)$, the incentive condition for types $C$ to communicate is
\begin{equation}\label{ICcom-F50}
	20 (\alpha(x(1,1)-x(1,0)+(1-\alpha)(x(0,1)-x(0,0))))\geq 3.	
\end{equation}

There are many solutions satisfying this condition as well as \eqref{CWC-BR11}-\eqref{CWC-BR00}.

\paragraph{``Calibration''.} In the experiment, $\alpha=0.50$, $\beta\approx 0$, $x(1,1)=1$, $x(0,1)=80\%, x(1,0)=70\%, x(0,0)=31\%$. The only theoretical prediction that roughly matches the order of magnitudes of these values are $x(1,1)=x(1,0)=x(0,1)=1$ and $x(0,0)=\frac{5}{18}\frac{2-\alpha}{1-\alpha}$. For these values, the left hand side of \eqref{ICcom-F50} is $\frac{5}{18}(2-\alpha)$, which can be equal to $\frac{3}{20}$ when $\alpha=\frac{41}{25}$, clearly impossible. (Alternatively, if $\alpha=\frac{1}{2}$, the left hand side has value $\frac{15}{36}$, that is three times smaller than $\frac{3}{20}$.)

We conclude that a standard model of signalling without cost-of-lying cannot explain the experimental data.

\bibliographystyle{agsm} 
\bibliography{}

\end{document}
